\documentclass[12pt,a4paper]{mwart}

\usepackage{lmodern}
\usepackage[T1]{polski}
\usepackage[utf8]{inputenc}

\usepackage[a4paper,
            tmargin=2cm,
            bmargin=2cm,
            lmargin=2cm,
            rmargin=2cm,
            bindingoffset=0cm]{geometry}

\usepackage{tocloft}
\usepackage{hyperref}

\usepackage{amsmath}
\usepackage{amssymb}
\usepackage{siunitx}

\usepackage{graphicx}
\usepackage{subfig}
\usepackage{float}
\usepackage{booktabs}


\hypersetup{
    colorlinks,
    citecolor=black,
    filecolor=black,
    linkcolor=black,
    urlcolor=black
}

\begin{document}

\title{Potencjał elektryczny}
\author{Szymon Idec}
\date{\today}
\maketitle

\section{Sformułowanie silne}
\begin{equation*}
    \begin{gathered}
    \frac{\mathrm{d}^2\phi}{\mathrm{d}x^2} = -\frac{\rho}{\epsilon_r} \\
    \phi'(0) + \phi(0) = 5 \\
    \phi(3) = 2 \\
    \rho = 1 \\
    \epsilon_r(x) =
        \begin{cases}
            10 & \text{dla $x \in [0, 1]$} \\
            5  & \text{dla $x \in (1, 2]$} \\
            1  & \text{dla $x \in (2, 3]$} \\
        \end{cases}
    \end{gathered}
\end{equation*}

Gdzie poszukiwana funkcja to $\phi(x)$.
\begin{equation*}
    [0, 3] \ni x \rightarrow \phi(x) \in \mathbb{R}
\end{equation*}

\newpage

\section{Warunki brzegowe}
Na prawym brzegu mamy niezerowy warunek Dirichleta. Szukamy więc rozwiązań postaci $\phi = w + \widetilde{u}$.
Przyjmujemy $\widetilde{u} = 2e_n$

Na lewym brzegu mamy warunek Robina. Otrzymujemy z niego $\phi'(0) = 5 - \phi(0)$.


\section{Sformułowanie słabe}

\begin{align*}
    \phi'' &= -\frac{\rho}{\epsilon_r} \\
    \phi''v &= -\frac{\rho}{\epsilon_r}v \\
    \underbrace{\int_0^3\phi''v\mathrm{d}x}_\text{całk. przez części} &= -\int_0^3\frac{\rho}{\epsilon_r}v\mathrm{d}x \\
    \phi'v'\big|_0^3 - \int_0^3\phi'v'\mathrm{d}x &= -\frac{1}{10}\int_0^1v\mathrm{d}x -\frac{1}{5}\int_1^2v\mathrm{d}x -\int_2^3v\mathrm{d}x \\
    \phi'(3)\underbrace{v(3)}_{0} - \phi'(0)v(0) - \int_0^3\phi'v'\mathrm{d}x &= -\frac{1}{10}\int_0^1v\mathrm{d}x -\frac{1}{5}\int_1^2v\mathrm{d}x -\int_2^3v\mathrm{d}x \\
    - (5-\phi(0))v(0) - \int_0^3\phi'v'\mathrm{d}x &= -\frac{1}{10}\int_0^1v\mathrm{d}x -\frac{1}{5}\int_1^2v\mathrm{d}x -\int_2^3v\mathrm{d}x \\
    \underbrace{\phi(0)v(0) - \int_0^3\phi'v'\mathrm{d}x}_\text{$B(\phi, v)$} &= \underbrace{-\frac{1}{10}\int_0^1v\mathrm{d}x -\frac{1}{5}\int_1^2v\mathrm{d}x -\int_2^3v\mathrm{d}x + 5v(0)}_\text{$L(v)$} \\ 
\end{align*}

\begin{equation*}
    \begin{gathered}
        B(\phi, v) = B(w + \widetilde{u}, v) = B(w, v) + B(\widetilde{u}, v) \\
        B(w,v) = L(v) - B(\widetilde{u}, v)
    \end{gathered}
\end{equation*}

\section{Funkcje bazowe}
\begin{equation*}
    \begin{gathered}
    e_i(x) =
    \begin{cases}
        \frac{x - x_{i-1}}{x_i - x_{i-1}} & \text{dla $x\in(x_{i-1},x_i)$} \\
        \frac{x_{i+1} - x}{x_{i+1} - x_i} & \text{dla $x\in(x_i,x_{i+1})$}
    \end{cases} \\
    \end{gathered}
\end{equation*}

\section{Równanie w postaci macierzowej}

\begin{equation*}
    A \cdot U=B
\end{equation*}

\begin{equation*}
    \left[
    \begin{matrix}
        B(e_0, e_0) & B(e_1, e_0) & \cdots & B(e_{n-1}, e_0) \\
        B(e_0, e_1) & B(e_1, e_1) & \cdots & B(e_{n-1}, e_1) \\
        \vdots      & \vdots & \ddots & \vdots \\
        B(e_0, e_{n-1}) & B(e_1, e_{n-1}) & \cdots & B(e_{n-1}, e_{n-1})
    \end{matrix}
    \right]
    \cdot
    \left[
    \begin{matrix}
        u_0 \\
        u_1 \\
        \vdots \\
        u_{n-1}
    \end{matrix}
    \right]
    =
    \left[
    \begin{matrix}
        L(e_0) - B(\widetilde{u}, e_0) \\
        L(e_1) - B(\widetilde{u}, e_1) \\
        \vdots \\
        L(e_{n-1}) - B(\widetilde{u}, e_{n-1}))
    \end{matrix}
    \right]
\end{equation*}

\section{Uproszczenie wartości}
\begin{equation*}
    B(e_1, e_1) = B(e_2, e_2) = \dotsc = B(e_{n-1}, e_{n-1})
\end{equation*}

\begin{equation*}
    \begin{gathered}
        B(e_1, e_0) = B(e_2, e_1) = \dotsc = B(e_{n-1}, e_{n-2}) \\
        B(e_{i+1}, e_i) = B(e_i, e_{i+1}) \\
        i = 0, \dotsc, n-2
    \end{gathered}
\end{equation*}

Każda całka $B(e_a, e_b)$ gdzie $|a-b|>1$ będzie równa 0, ponieważ przedziały
funkcji bazowych się nie nakładają.

W macierzy B odejmujemy tylko w ostatnim wierszu $B(\widetilde{u}, e_i)$, ponieważ
w innych wierszach funkcje bazowe się nie nakładają ($\widetilde{u} = 2e_n$). 

\end{document}